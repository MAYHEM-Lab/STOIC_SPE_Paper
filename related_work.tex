We have explored an initial design and scheduler for STOIC
in~\cite{ref:stoic2020}.  The work herein extends this early work with a new
scheduling system and consideration of both individual and concurrent edge-cloud
placements. 

A significant body of work~\cite{ref:lowlatency, ref:bandwidth, ref:MAUI} has
explored low-latency geo-distributed data analytics and mobile-cloud offloading
-- which we take as inspiration for the STOIC design. One relevant approach is
federated learning~\cite{ref:federated}, by which a comprehensive model is
trained across heterogeneous edge devices or servers without exchanging local
data samples. Federated learning aims to address the security and networking
concerns by keeping the datasets local at devices, whereas STOIC intelligently
offloads jobs across multiple tiers of cloud infrastructure to further reduce
latency. 

In addition, STOIC targets IoT systems and leverages serverless
computing and GPUs. 
As such, other related work includes recent advances in machine learning
infrastructure, serverless computing, GPU accelerators, and container-based
orchestration services. \cite{ref:serverlessstep} and
\cite{ref:berkeleyserverless} conduct a comprehensive survey on serverless
computing including challenges and research opportunities. We share the same
viewpoint that the use of the serverless execution model will grow for online
training and inference applications. \cite{ref:deepserving} provides a prototype
for a deep learning model serving in a serverless platform.
\cite{ref:accelerated} provides another use case for accelerating serverless
functions by GPU virtualization in data centers. Unique in our work, STOIC
extends an existing serverless framework to support GPU acceleration and
distributed function placement across the edge and public clouds.
\cite{ref:evaluation} evaluates several serverless frameworks that use
Kubernetes to manage and orchestrate use of Linux containers. STOIC also
integrates Kubernetes for container orchestration, which is lightweight,
flexible, and developer-friendly.  We concur that Kubernetes is a promising
deployment infrastructure for serverless computing.  

Another relevant domain of related work is edge-to-cloud 
infrastructure enabling IoT device applications. 
\cite{ref:face} compares the processing time of face recognition 
between the edge device and smartphones.  It concludes that edge
devices perform comparably faster and scales better as the number of images
increases. We agree with this conclusion, and as such, we design STOIC to offload
image processing workloads to both edge clouds and public clouds.
\cite{ref:DDNN} proposes a distributed deep neural network that allows fast and
localized inference at the edge device using truncated layers of a neural
network. \cite{ref:cooperative} defines edge cloud offloading as a Markov
decision process (MDP) whose objective is to minimize the average processing
time per job. Based on this setting, it provides an approximate solution to
MDP with a one-step policy iteration. Similar to this approach,
\cite{ref:QoS-aware} proposes a Global Cluster Manager for orchestrating
network-intensive programs within Software-Defined Data Centers (SDDCs)
targeting high Quality of Service (QoS) and, further, \cite{ref:FQoS} classifies
available cloud deployment options by a stochastic Markov model, namely Formal
QoS Assurances Method (FoQoSAM), to optimize the automated offloading process.
Due to its practical utility, such a method can guarantee that 
QoS requirements are satisfied.

\cite{ref:trust} proposes a fog computing platform (DECENTER) and a trust
management architecture based on Smart Contracts. Related to this work,
\cite{ref:multichain} develops an architecture (HCL-BaFog) by the blockchain
functionality to share sensor data. Table~\ref{tab:related-work} summarizes the
properties of DECENTER, HCL-BaFog, and STOIC. These works are complementary to
STOIC and we are considering how to incorporate them into the system as part of
future work.

\begin{table}
\centering
\resizebox{390pt}{!}{
\scriptsize
\resizebox{\columnwidth}{!}{
\begin{tabular}{|c|c|c|c|} 
\hline
& \textbf{DECENTER} & \textbf{HCL-BaFog} & \textbf{STOIC} \\
\hline
\textbf{Node Selection} & FoQoSAM & MultiChain & Dynamic Feedback Loop \\
\hline
\textbf{Orchestration} & Kubernetes & Docker Swarm & Kubeless \\
\hline
\textbf{Quality of Service} & latency/throughput/availability & latency/availability & latency/availability \\
\hline
\textbf{Trust Mgmt} & Smart Contracts & Blockchain & Nautilus \\ 
\hline
\textbf{Application} & Video Streaming & Sensor Data Sharing & Image Recognition \\
\hline
\end{tabular}
}}
\caption{ The comparison table of DECENTER, HCL-BaFog and STOIC.
\label{tab:related-work}}
\end{table}

Also complimentary to STOIC, are tracing, testing, repair, and profiling tools
(which STOIC can leverage) for serverless systems. Multiple works track causal
dependencies across distributed serverless deployments for use in optimization,
placement, and data
repair~\cite{ref:repairdata,deptracing19,gammaray17,aws-xray}.
FaaSProfiler~\cite{ref:profile} integrates testing and profiling 
within a FaaS platform. \cite{ref:security} proposes a security solution that
applies reinforcement learning (RL) to provide secure offloading to the edge
nodes to prevent jamming attacks. These related systems can be combined with
STOIC to provide a robust serverless ecosystem for distributed IoT devices.

